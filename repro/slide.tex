\chapter{Reproducible Research}\alabel{chap:repro} \ddisc{20}
\quoteit{Disconfirmation bias: giving expected results a relatively
  free pass but rigorously checking non-intuitive results}{\citet{nuz15how}}

An excellent article on how to do reproducible research is
\cite{mun17man} for which the pdf file is openly available.  The link
to an excellent video by Garrett Grolemund is on the right. \movie{https://resources.rstudio.com/rstudio-conf-2019/r-markdown-the-bigger-picture}

\href{https://journals.sagepub.com/doi/full/10.1177/0146167217729162}{this article} by Gelman discussed the role that statistical ``significance'' plays in non-reproducible research.

\section{Non-reproducible Research}
\soundm{rr-1}
\bi
\item Misunderstanding statistics
\item ``Investigator'' moving the target
\item Lack of a blinded analytic plan
\item Tweaking instrumentation / removing ``outliers''
\item Floating definitions of response variables
\item Pre-statistician ``normalization'' of data and background subtraction 
\item Poorly studied high-dimensional feature selection
\item Programming errors
\item Lack of documentation
\item Failing to script multiple-step procedures
 \bi
 \item using spreadsheets and other interactive approaches for data
 manipulation
 \ei
\item Copying and pasting results into manuscripts
\item Insufficient detail in scientific articles
\item No audit trail
\ei

\section{General Importance of Sound Methodology}
\soundm{rr-2}
\subsection{Translation of Research Evidence from Animals to Humans}
\bi
\item Screened articles having preventive or therapeutic intervention
  in in vivo animal model, $> 500$ citations (\citet{hac06tra}) 
\item 76 ``positive'' studies identified
\item Median 14 years for potential translation
\item 37 judged to have good methodological quality (flat over time)
\item 28 of 76 replicated in human randomized trials; 34 remain untested
\item $\uparrow$ 10\% methodology score $\uparrow$ odds of
replication $\times$ 1.28 (0.95 CL 0.97--1.69)
\item Dose-response demonstrations: $\uparrow$ odds $\times$ 3.3 (1.1--10.1)
\ei
Note: The article misinterpreted $P$-values

\subsection{Other Problems}
\soundm{rr-3}
\bi
\item Rhine and ESP: ``the student's extra-sensory perception ability
has gone through a marked decline''
\item Floating definitions of $X$ or $Y$: association between physical
  symmetry and mating behavior; acupuncture
\item Selective reporting and publication bias
\item Journals seek confirming rather than conflicting data
\item Damage caused by hypothesis tests and cutoffs
\item Ioannidis: $\frac{1}{3}$ of articles in \emph{Nature} never get
  \textbf{cited}, let alone replicated 
\item Biologic and lab variability
\item Unquestioning acceptance of research by the ``famous''
  \bi
  \item Weak coupling ratio exhibited by decaying neutrons fell by 10
    SDs from 1969--2001
  \ei
\ei

\subsection{What's Gone Wrong with Omics \& Biomarkers?}
\soundm{rr-4}
\bi
\item Gene expression-based
prognostic signatures in lung cancer: Ready for clinical use? (\citet{sub10gen})
\item NSCLC gene expression studies 2002--2009, $n \geq 50$
\item 16 studies found
\item Scored on appropriateness of protocol, stat validation, medical
utility
\item Average quality score: 3.1 of 7 points
\item No study showed prediction improvement over known risk factors;
  many failed to validate
\item Most studies did not even consider factors in guidelines
 \bi
 \item Completeness of resection only considered in 7
 \item Similar for tumor size
 \item Some only adjusted for age and sex
 \ei
\ei

\subsection{Failure of Replication in Preclinical Cancer Research}
\soundm{rr-5}
\bi
\item Scientists at Amgen tried to confirm published findings related
  to a line of research, before launching development
\item Identified 53 `landmark' studies
\item Scientific findings confirmed in only 6 studies
\item Non-reproduced articles cited far more frequently than
  reproduced articles
\ei
Begley CG, Ellis LM: Raise standards for preclinical cancer
  research.\\Nature 483:531-533; 2012

\subsubsection{Natural History of New Fields}
\quoteit{
Each new field has a rapid exponential growth of its literature over
5--8 years (``new field phase''), followed by an ``established field''
phase when growth rates are more modest, and then an ``over-maturity''
phase, where the rates of growth are similar to the growth of the
scientific literature at large or even smaller.  There is a parallel
in the spread of an infectious epidemic that emerges rapidly and gets
established when a large number of scientists (and articles) are
infected with these concepts.  Then momentum decreases, although many
scientists remain infected and continue to work on this field.  New
omics infections continuously arise in the scientific community.}{\citet{ion10exp}}

\includegraphics[width=\textwidth]{nytimes.png}\soundm{rr-6}

\section{System Forces}
\includegraphics[width=\textwidth]{../slinks/sysqual.pdf}

\section{Strong Inference}
\soundm{rr-7}
\quoteit{Cognitive biases are hitting the accelerator of science: the
  process spotting potentially important scientific relationships.
  Countering those biases comes down to strengthening the `brake': the
  ability to slow down, be sceptical of findings and eliminate false
  positives and dead ends.}{\citet{nuz15how}}

\includegraphics[width=\textwidth]{pla64str.png}

\citet{pla64str}

\bi
\item Devise alternative hypotheses
\item Devise an experiment with alternative possible outcomes each of
which will exclude a hypothesis
\item Carry out the experiment
\item Repeat
\item Regular, explicit use of alternative hypotheses \& sharp
exclusions $\rightarrow$ rapid \& powerful progress
\item ``Our conclusions \ldots might be invalid if \ldots (i) \ldots
(ii) \ldots (iii) \ldots We shall describe experiments which eliminate
these alternatives.''\cite{pla64str}
\ei


\section{Pre-Specified Analytic Plans}
\soundm{rr-8}
\quoteit{I have enormous flexibility in how I analyze my data and what
  I choose to report.  This creates a conflict of interest.  The only
  way to avoid this is for me to tie my hands in advance.
  Precommitment to my analysis and reporting plan mitigates the
  influence of these cognitive biases.}{Brian Nosek, Center for Open
  Science\cite{nuz15how}}

\bi
\item Long the norm in multi-center RCTs
\item Needs to be so in \textbf{all} fields of research using data to
draw inferences~\cite{rub07des}
\item Front-load planning with investigator
 \bi
 \item too many temptations later once see results (e.g., $P=0.0501$)
 \ei
\item SAP is signed, dated, filed
\item Pre-specification of reasons for exceptions, with exceptions
documented (when, why, what)
\item Becoming a policy in VU Biostatistics
\ei


\section{Summary}
\soundm{rr-9}
Methodologic experts have much to offer:
\bi
\item Biostatisticians and clinical epidemiologists play important
roles in
 \bi
 \item assessing the needed information content for a given problem
   complexity
 \item minimizing bias
 \item maximizing reproducibility
 \ei
\item For more information see:
\bi
\item \co{ctspedia.org}
\item \co{reproducibleresearch.net}
\item \co{groups.google.com/group/reproducible-research}
\ei
\ei

\section{Software}\alabel{sec:repro-software}
\soundm{rr-10}
\subsection{Goals of Reproducible Analysis/Reporting}
\bi
\item Be able to reproduce your own results
\item Allow others to reproduce your results

  \quoteit{Time turns each one of us into another person, and by
    making effort to communicate with strangers, we help ourselves to
    communicate with our future selves.}{Schwab and Claerbout}
\item Reproduce an entire report, manuscript, dissertation, book with
a single system command when changes occur in:
 \bi
 \item operating system, stat software, graphics engines, source data, derived
 variables, analysis, interpretation
 \ei
\item Save time
\item Provide the ultimate documentation of work done for a paper
\ei
See \url{http://hbiostat.org/rr}

\subsection{History of Literate Programming}
\soundm{rr-11}
\bi
\item Donald Knuth found his own programming to be sub-optimal
\item Reasons for programming attack not documented in code; code hard to read
\item Invented \textbf{literate programming} in 1984
 \bi
 \item mix code with documentation in same file
 \item ``pretty printing'' customized to each, using \TeX
 \item not covered here: a new way of programming
 \ei
\item Knuth invented the \co{noweb} system for combining two types of
 information in one file
 \bi
 \item \emph{weaving} to separate non-program code
 \item \emph{tangling} to separate program code
 \ei
\ei
See \url{http://www.ctan.org/tex-archive/help/LitProg-FAQ}
\bi
\item Leslie Lamport made \TeX\ easier to use with a comprehensive
 macro package \LaTeX\ in 1986
\item Allows the writer to concern herself with structures of ideas,
not typesetting
\item \LaTeX\ is easily modifiable by users: new macros,
variables, \emph{if-then} structures, executing system commands (Perl,
etc.), drawing commands, etc.
\item S system: Chambers, Becker, Wilks of Bell Labs, 1976
\item \R\ created by Ihaka and Gentleman in 1993, grew partly as a response
 to non-availability of S-Plus on Linux and Mac
\item Friedrich Leisch developed Sweave in 2002
\item Yihui Xie developed \co{knitr} in 2011
\ei

\begin{figure}\begin{center}{\smaller[-2] A Bad Alternative to \co{knitr}}\\%
\includegraphics[width=.6\textwidth]{excel.png}
\end{center}\end{figure}

\subsection{\co{knitr} Approach}
\soundm{rr-12}
\bi
\item \co{knitr} is an \R\ package on CRAN
\item Uses \co{noweb} and an \co{sweave} style in \LaTeX; see
\url{yihui.name/knitr}, \cite{knitrbook}, \url{http://yihui.github.com/knitr}
\item \co{knitr} also works with \co{Markdown} and other languages
\item \co{knitr} is tightly integrated into \co{RStudio}
\item \emph{Insertions} are a major component
 \bi
 \item \R\ printout after code chunk producing the output; plain tables
 \item single \co{pdf} or \co{postscript} graphic after chunk,
 generates \LaTeX\ \co{includegraphics} command
 \item direct insertion of \LaTeX\ code produced by \R\ functions
 \item computed values inserted outside of code chunks
 \ei
\item Major advantages over Microsoft Word: composition time, batch
 mode, easily maintained scripts, beauty
\item \co{knitr} produces self-documenting reports with nice graphics,
to be given to clients
 \bi
 \item showing code demonstrates you are not doing ``pushbutton'' research
 \ei
\ei

\subsection{\co{knitr} Features}
\soundm{rr-13}
\bi
\item \R\ code set off by lines containing only \verb|<<>>=|
\item \LaTeX\ text starts with a line containing only \verb|@|
\item \co{knitr} senses when a chunk produces a graphic (even without
  \emph{print()} and
  automatically includes the graphic in \LaTeX
\item All other lines sent to \LaTeX\ verbatim, \R\ code and output sent to
  \LaTeX\ by default but this can easily be overridden
\item Can specify that a chunk produces markup that is directly
  typeset; this is how complex \LaTeX\ tables generated by \R
\item Can include calculated variables directly in sentences, e.g.\\
 \verb|And the final answer is 3|.
 will produce ``And the final answer is 3.''
\item Easy to customize chunk options and add advanced features such
  as automatically creating a \LaTeX\ \co{figure} environment if a
  caption is given in the chunk header
\item Setup for advanced features, including code pretty-printing,
  shown at \url{https://biostat.app.vumr.org/KnitrHowto}
\item Simplified interface to \co{tikz} graphics
\item Simplified implementation of caching
\item More automatic pretty--printing; support
  for \LaTeX\ \co{listings} package built--in
\ei
See  \url{http://hbiostat.org/rr/index.html#template}

\subsubsection{Summary}
{\smaller[2]
Much of research that uses data analysis is not reproducible.
This can be for a variety of reasons, the most major one being poor
design and poor science.  Other causes include tweaking of
instrumentation, the use of poorly studied high-dimensional feature
selection algorithms, programming errors, lack of adequate
documentation of what was done, too much copy and paste of results
into manuscripts, and the use of spreadsheets and other interactive
data manipulation and analysis tools that do not provide a usable
audit trail of how results were obtained.  Even when a research
journal allows the authors the ``luxury'' of having space to describe
their methods, such text can never be specific enough for readers to
exactly reproduce what was done.  All too often, the authors
themselves are not able to reproduce their own results.  Being able to
reproduce an entire report or manuscript by issuing a single operating
system command when any element of the data change, the statistical
computing system is updated, graphics engines are improved, or the
approach to analysis is improved, is also a major time saver.}

{\smaller[2]
It has been said that the analysis code provides the ultimate
documentation of the ``what, when, and how'' for data analyses.  Eminent
computer scientist Donald Knuth invented literate programming in 1984
to provide programmers with the ability to mix code with documentation
in the same file, with ``pretty printing'' customized to each.
Lamport's \LaTeX, an offshoot of Knuth's \TeX\ typesetting system, became
a prime tool for printing beautiful program documentation and manuals.
When Friedrich Leisch developed Sweave in 2002, Knuth's literate
programming model exploded onto the statistical computing scene with a
highly functional and easy to use coding standard using \R\ and \LaTeX\
and for which the Emacs text editor has special dual editing modes
using ESS.  This approach has now been extended to other computing
systems and to word processors.  Using \R\ with \LaTeX\ to construct
reproducible statistical reports remains the most flexible approach
and yields the most beautiful reports, while using only free software.
One of the advantages of this platform is that there are many
high-level \R\ functions for producing \LaTeX\ markup code directly, and
the output of these functions are easily directly to the \LaTeX\ output
stream created by \co{knitr}.}

\section{Further Reading}
An excellent book is \citet{irr}.  See also
\bi
\item \url{https://github.com/SISBID/Module3}: course by Baggerly and Broman
\item \url{reproducibleresearch.net}
\item \url{cran.r-project.org/web/views/ReproducibleResearch.html}
\item \url{www.nature.com/nature/focus/reproducibility}
\item \url{hbiostat.org/rr}
\item \url{groups.google.com/forum/#!forum/reproducible-research}
\item \url{resources.rstudio.com/rstudio-conf-2019/r-markdown-the-bigger-picture}
\ei

    
