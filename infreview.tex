\chapter{Statistical Inference Review}
\bi
\item Emphasize confidence limits, which can be computed from adjusted
  or unadjusted analyses, with or without taking into account multiple
  comparisons
\item $P$-values can accompany CLs if formal hypothesis testing needed
\item When possible construct $P$-values to be consistent with how CLs
  are computed
\ei

\section{Types of Analyses}
\bi
\item Except for one-sample tests, all tests can be thought of as
  testing for an association between at least one variable with at
  least one other variable
\item Testing for group differences is the same as testing for
  association between group and response
\item Testing for association between two continuous variables can be
  done using correlation (especially for unadjusted analysis) or
  regression methods; in simple cases the two are equivalent
\item Testing for association between group and outcome, when there
  are more than 2 groups which are not in some solid
  order\footnote{The dose of a drug or the severity of pain are
    examples of ordered variables.} means comparing a summary of the
  response between $k$ groups, sometimes in a pairwise fashion
\ei

\section{Covariable-Unadjusted Analyses}
Appropriate when
\bi
\item Only interested in assessing the relationship between a single
   $X$ and the response, or
\item Treatments are randomized and there are no strong prognostic
   factors that are measureable
\item Study is observational and variables capturing confounding are
   unavailable (place strong caveats in the paper)
\ei
See ~\ref{chap:ancova}.   %%% ?? TODO

\subsection{Analyzing Paired Responses}
\begin{center}\begin{tabular}{lll} \hline
Type of Response & Recommended Test & Most Frequent Test \\ \hline
binary     & McNemar & McNemar \\
continuous & Wilcoxon signed-rank & paired $t$-test \\ \hline
\end{tabular}\end{center}

\subsection{Comparing Two Groups}
\begin{minipage}{\textwidth}
\begin{center}\begin{tabular}{lll} \hline
Type of Response & Recommended Test & Most Frequent Test \\ \hline
binary     & $2\times 2 \chi^{2}$ & $\chi^{2}$, Fisher's exact test \\
ordinal    & Wilcoxon 2-sample & Wilcoxon 2-sample \\
continuous & Wilcoxon 2-sample & 2-sample $t$-test \\
time to event\footnote{The response variable may be
  right-censored, which happens if the subject ceased being followed
     before having the event.  The value of the response variable, for
   example, for a subject followed 2 years without having the event is
   2+.} & Cox model\footnote{If the treatment is expected to have more
   early effect with the effect lessening over time, an accelerated
   failure time model such as the lognormal model is recommended.}&
 log-rank\footnote{The log-rank is a special case 
   of the Cox model.  The Cox model provides slightly more accurate
   $P$-values than the $\chi^2$ statistic from the log-rank test.} \\
 \hline
\end{tabular}\end{center}\end{minipage}

\subsection{Comparing $>2$ Groups}
\begin{center}\begin{tabular}{lll} \hline
Type of Response & Recommended Test & Most Frequent Test \\ \hline
binary     & $r\times 2 \chi^{2}$ & $\chi^{2}$, Fisher's exact test \\
ordinal    & Kruskal-Wallis & Kruskal-Wallis \\
continuous & Kruskal-Wallis & ANOVA \\
time to event & Cox model & log-rank \\ \hline
\end{tabular}\end{center}

\subsection{Correlating Two Continuous Variables}
Recommended: Spearman $\rho$ \\
Most frequently seen: Pearson $r$

\section{Covariable-Adjusted Analyses}
\bi
\item To adjust for imbalances in prognostic factors in an
  observational study or for strong patient heterogeneity in a
  randomized study
\item Analysis of covariance is preferred over stratification,
  especially if continuous adjustment variables are present or there
  are many adjustment variables
 \bi
 \item Continuous response: multiple linear regression with
   appropriate transformation of $Y$
 \item Binary response: binary logistic regression model
 \item Ordinal response: proportional odds ordinal logistic regression
   model
 \item Time to event response, possibly right-censored: 
  \bi
  \item chronic disease: Cox proportional hazards model
  \item acute disease: accelerated failure time model
  \ei
 \ei
\ei
