% Converted obsVar.pdf using newocr.com
% Usage: knitr slide


\chapter{Analysis of Observer Variability and Measurement
  Agreement}\blfootnote{FE Harrell, 1987}\alabel{chap:obsvar}
\section{Intra- and Inter-observer Disagreement}
Before using a measurement instrument or diagnostic technique
routinely, a researcher may wish to quantify the extent to which two
determinations of the measurement, made by two different observers
or measurement devices, disagree (inter-observer variability).  She
may also wish to quantify the repeatability of one observer in making
the measurement at different times (intra-observer variability). To
make these assessments, she has each observer make the measurement
for each of a number of experimental units (e.g., subjects).

The measurements being analyzed may be continuous, ordinal, or binary
(yes/no). Ordinal measurements must be coded such that distances between
values reflects the relative importance of disagreement. For example, if a
measurement has the values 1, 2, 3 for poor, fair, good, it is assumed that
``good'' is as different from ``fair'' as ``fair'' is from
``poor''. if this is  not the case, a different coding should be used,
such as coding 0 for ``poor'' if poor should be twice as far from
``fair'' as ``fair'' is from ``good''. Measurements that are yes/no or
positive/negative should be coded as 1 or 0. The reason for this will
be seen below.  

There are many statistical methods for quantifying inter- and
intra-observer variability.  Correlation coefficients are frequently reported,
but a perfect correlation can result even when the measurements disagree by a
factor of 10.  Variance components analysis and intra-class
correlation are often used, but these make many assumptions, do not
handle missing data very well, and are difficult to interpret. Some
analysts, in assessing 
inter-observer agreement when each observer makes several determinations,
compute differences between the average determinations for each observer.
This method clearly yields a biased measurement of inter-observer agreement
because it cancels the intra-observer variability.

A general and descriptive method for assessing observer variability
will now be presented. The methods uses a general type of statistic
called the $U$ statistic, invented by
Hoeffding~\cite{hoe48cla}\footnote{The Wilcoxon test and the $c$-index
  are other examples of $U$ statistics.}. For definiteness, an
analysis for 3 observers and 2 readings per observer will be
shown. When designing such a study, the researcher should remember
that the number of experximental units is usually the critical factor
in determining the precision of estimates. There is not much to gain
from having each observer make more than a few readings or from having
30 observers in the study (although if few observers are used, these
are assumed to be ``typical'' observers).

The intra-observer disagreement for a single subject or unit is defined
as the average of the intra-observer absolute measurement differences. In
other words, intra-observer disagreement is the average absolute difference
between any two measurements from the same observer. The inter-observer
disagreement for one unit is defined as the average absolute difference
between any two readings from different observers. Disagreement measures are
computed separately for each unit and combined over units (by taking the mean
or median for example) to get an overall summary measure.  Units
having more readings get more weight.  When a reading is
missing, that reading does not enter into any calculation and the denominator
used in finding the mean disagreement is reduced by one.

Suppose that for one patient, observers A, B, and C make the following
determinations on two separate occasions, all on the
same patient:
\begin{center}\begin{tabular}{ccc}
A & B & C \\
5,7 & 8,5 & 6,7
\end{tabular}\end{center}
For that patient, the mean intra-observer difference is $(|5-7| +
|8-5| + |6-7|)/3 = \frac{2+3+1}{3} = 2$. The mean inter-observer difference is
$(|5-8| + |5-5| + |5-6|+ |5-7| + |7-8| + |7-5| + |7-6| + |7-7|+ |8-6|
+ |8-7| + |5-6| + |5-7|)/12 = (3+0+1+2+1+2+1+0+2+1+1+2)/12 =
\frac{16}{12} = 1.33$. 
If the first reading for observer A were unobtainable, the mean
intra-observer difference for that patient would be $(|8-5| + |6-7|)/2
= \frac{3+1}{2} = 2$ and the mean inter-observer difference would be $(|7-8|
+ |7-5| + |7-6| + |7-7| + |8-6| + |8-7| + |5-6| + |5-7|)/8 =
(1+2+1+0+2+1+1+2)/8 = \frac{10}{8} = 1.25$. 

The computations are carried out in like manner for each patient and
summarized as follows:

\begin{center}\begin{tabular}{ccc}
Patient & Intra-observer & Inter-observer\\
        & Difference     & Difference\\ \hline
1 & 2.00 & 1.33\\
2 & 1.00 & 3.50\\
3 & 1.50 & 2.66\\
. & .    & .\\
. & .    & .\\
$n$ & . & . \\ \hline
\textbf{Overall Average} (or median) & 1.77 & 2.23\\
$Q_{1}$ & 0.30 & 0.38\\
$Q_{3}$ & 2.15 & 2.84\\
\end{tabular}\end{center}

Here is an example using \R\ to compute mean inter- and
intra-observer absolute differences for 4 subjects each assessed
twice by each of 3 observers.  The first subject consists of data
above.  The calculations are first done for the first subject alone,
to check against computations above.
\begin{Schunk}
\begin{Sinput}
d <- expand.grid(rep=1:2, observer=c('A','B','C'), subject=1:4)
d$y <- c(5,7, 8,5, 6,7,
         7,6, 8,6, 9,7,
         7,5, 4,6, 10,11,
         7,6, 5,6, 9,8)
d
\end{Sinput}
\begin{Soutput}
   rep observer subject  y
1    1        A       1  5
2    2        A       1  7
3    1        B       1  8
4    2        B       1  5
5    1        C       1  6
6    2        C       1  7
7    1        A       2  7
8    2        A       2  6
9    1        B       2  8
10   2        B       2  6
11   1        C       2  9
12   2        C       2  7
13   1        A       3  7
14   2        A       3  5
15   1        B       3  4
16   2        B       3  6
17   1        C       3 10
18   2        C       3 11
19   1        A       4  7
20   2        A       4  6
21   1        B       4  5
22   2        B       4  6
23   1        C       4  9
24   2        C       4  8
\end{Soutput}
\begin{Sinput}
# Function to compute mean absolute discrepancies
mad <- function(y, obs, subj) {
  nintra <- ninter <- sumintra <- suminter <- 0
  n <- length(y)
  for(i in 1 : (n - 1)) {
    for(j in (i + 1) : n) {
      if(subj[i] == subj[j]) {
        dif <- abs(y[i] - y[j])
        if(! is.na(dif)) {
          if(obs[i] == obs[j]) {
            nintra   <- nintra + 1
            sumintra <- sumintra + dif
          }
          else {
            ninter   <- ninter + 1
            suminter <- suminter + dif
          }
        }
      }
    }
  }
  c(nintra=nintra, intra=sumintra / nintra,
    ninter=ninter, inter=suminter / ninter)
}
  
# Compute statistics for first subject
with(subset(d, subject == 1), mad(y, observer, subject))
\end{Sinput}
\begin{Soutput}
   nintra     intra    ninter     inter 
 3.000000  2.000000 12.000000  1.333333 
\end{Soutput}
\begin{Sinput}
# Compute for all subjects
with(d, mad(y, observer, subject))
\end{Sinput}
\begin{Soutput}
   nintra     intra    ninter     inter 
12.000000  1.583333 48.000000  2.125000 
\end{Soutput}
\end{Schunk}
Zhouwen Liu in the Vanderbilt Department of Biostatistics has
developed much more general purpose software for this in \R.  Its web
pages are
\url{http://biostat.app.vumc.org/AnalysisOfObserverVariability}
and \url{https://github.com/harrelfe/rscripts}.
The following example loads the source code and runs the above
example.  The \R\ functions implement bootstrap nonparametric
percentile confidence limits for mean absolute discrepency measures.
\begin{Schunk}
\begin{Sinput}
require(Hmisc)
getRs('observerVariability.r', put='source')
\end{Sinput}
\begin{Soutput}
YOU ARE RUNNING A DEMO VERSION 3_2 
\end{Soutput}
\begin{Sinput}
with(d, {
  intra <- intraVar(subject, observer, y)
  print(intra)
  summary(intra)
  set.seed(2)
  b=bootStrap(intra, by = 'subject', times=1000)
  # Get 0.95 CL for mean absolute intra-observer difference
  print(quantile(b, c(0.025, 0.975)))
  inter <- interVar(subject, observer, y)
  print(inter)
  summary(inter)
  b <- bootStrap(inter, by = 'subject', times=1000)
  # Get 0.95 CL for mean absolute inter-observer difference
  print(quantile(b, c(0.025, 0.975)))
})
\end{Sinput}
\begin{Soutput}
Intra-Variability 

Measures by Subjects and Raters
   subject rater variability N
1        1     A           2 1
2        1     B           3 1
3        1     C           1 1
4        2     A           1 1
5        2     B           2 1
6        2     C           2 1
7        3     A           2 1
8        3     B           2 1
9        3     C           1 1
10       4     A           1 1
11       4     B           1 1
12       4     C           1 1

Measures by Subjects
  subject variability N
1       1    2.000000 3
2       2    1.666667 3
3       3    1.666667 3
4       4    1.000000 3

Measures by Raters
  subject variability N
1       A        1.50 4
2       B        2.00 4
3       C        1.25 4
Intra-variability summary:

Measures by Subjects and Raters
Variability mean:  1.583333 
Variability median:  1.5 
Variability range:  1 3 
Variability quantile: 
0%:  1   25%:  1   50%:  1.5   75%:  2   100%:  3 

Measures by Subjects
Variability mean:  1.583333 
Variability median:  1.666667 
Variability range:  1 2 
Variability quantile: 
0%:  1   25%:  1.5   50%:  1.666667   75%:  1.75   100%:  2 

Measures by Raters
Variability mean:  1.583333 
Variability median:  1.5 
Variability range:  1.25 2 
Variability quantile: 
0%:  1.25   25%:  1.375   50%:  1.5   75%:  1.75   100%:  2 
    2.5%    97.5% 
1.166667 1.916667 
Input object size:	 3192 bytes;	 9 variables	 12 observations
Renamed variable	 rater 	to rater1 
Renamed variable	 disagreement 	to variability 
Dropped variables	diff,std,auxA1,auxA2
New object size:	1592 bytes;	5 variables	12 observations
Inter-Variability 

Measures for All Pairs of Raters
   subject rater1 rater2 variability N
1        1      1      2         1.5 4
5        2      1      2         1.0 4
9        3      1      2         1.5 4
13       4      1      2         1.0 4
17       1      1      3         1.0 4
21       2      1      3         1.5 4
25       3      1      3         4.5 4
29       4      1      3         2.0 4
33       1      2      3         1.5 4
37       2      2      3         1.5 4
41       3      2      3         5.5 4
45       4      2      3         3.0 4

Measures by Rater Pairs
  rater1 rater2 variability  N
1      1      2       1.250 16
2      1      3       2.250 16
3      2      3       2.875 16

Measures by Subjects
  subject variability  N
1       1    1.333333 12
2       2    1.333333 12
3       3    3.833333 12
4       4    2.000000 12
Inter-variability summary:
Measures for All Pairs of Raters
Variability mean:  2.125 
Variability median:  1.5 
Variability range:  1 5.5 
Variability quantile: 
0%:  1   25%:  1.375   50%:  1.5   75%:  2.25   100%:  5.5 

Measures by Rater Pairs only
Variability mean:  2.125 
Variability median:  2.25 
Variability range:  1.25 2.875 
Variability quantile: 
0%:  1.25   25%:  1.75   50%:  2.25   75%:  2.5625   100%:  2.875 

Measures by Subjects
Variability mean:  2.125 
Variability median:  1.666667 
Variability range:  1.333333 3.833333 
Variability quantile: 
0%:  1.333333   25%:  1.333333   50%:  1.666667   75%:  2.458333   100%:  3.833333 
    2.5%    97.5% 
1.333333 3.208333 
\end{Soutput}
\begin{Sinput}
# To load a demo file into an RStudio script editor window, type
# getRs('observerVariability_example.r')
\end{Sinput}
\end{Schunk}
From the above output, the 0.95 CL for the mean absolute
intra-observer difference is $[1.17, 1.92]$ and is $[1.33, 3.21]$ for the
inter-observer difference.  The bootstrap confidence intervals use the
cluster bootstrap to account for correlations of multiple readings
from the same subject.

When the measurement of interest is a yes/no determination such as
presence or absence of a disease these difference statistics are generalizations of the fraction of units in which there is exact agreement in the yes/no
determination, when the absolute differences are summarized by averaging. To
see this, consider the following data with only one observer:
\begin{center}\begin{tabular}{ccccc}
Patient & Determinations & $D_{1}, D_{2}$ & Agreement? & $|D_{1} -D_{2}|$\\ \hline
1 &Y Y& 1 1&Y&0\\
2 &Y N& 1 0&N&1\\
3 &N Y& 0 1&N&1\\
4 &N N& 0 0&Y&0\\
5 &N N& 0 0&Y&0\\
6 &Y N& 1 0&N&1\\ \hline
\end{tabular}\end{center}
The average $|D_{1} - D_{2}|$ is $\frac{3}{6} = 0.5$ which is equal to the
proportion of cases in which the two readings disagree.

An advantage of this method of summarizing observer differences is that
the investigator can judge what is an acceptable difference and he can relate
this directly to the summary disagreement statistic.

\section{Comparison of Measurements with a Standard}
When the true measurement is known for each unit (or the true diagnosis
is known for each patient), similar calculations can he used to quantify the
extent of errors in the measurements. For each unit, the average (over
observers) difference from the true value is computed and these differences
are summarized over the units. For example, if for unit \#1 observer A measures 5 and 7, observer 8 measured 8 and 5, and the true value is 6, the
average absolute error is $(|5-6|+|7-6|+ |8-6|+|5-6|)/4 = \frac{1+1+2+1}{4}
= \frac{5}{4} = 1.25$

\section{Special Case: Assessing Agreement In Two Binary Variables}
\subsection{Measuring Agreement Between Two Observers}
Suppose that each of $n$ patients undergoes two diagnostic tests that can
yield only the values positive and negative. The data can be summarized in
the following frequency table.
\begin{center}\begin{tabular}{cccc}
& & Test 2 & \\
      &   & + - &  \\
Test 1    & + & a b & g\\
          & - & c d & h\\
          &   & e f & n
\end{tabular}\end{center}
An estimate of the probability that the two tests agree is 
$p_{A}=\frac{a+d}{n}$. An approximate 0.95
confidence interval for the true probability is derived from
$p_{A} \pm 1.96 \sqrt{p_{A} (1 - p_{A})/n}$~\footnote{A more accurate confidence 
interval can be obtained using Wilson's method as provided by the \R\
\co{Hmisc} package \co{binconf} function.}
If the disease being tested is very rare or very common, the two tests
will agree with high probability by chance alone. The $\kappa$ statistic is one 
way to measure agreement that is corrected for chance agreement.
\begin{equation}
\kappa = \frac{p_{A} - p_{C}}{1 - p_{C}}
\end{equation}
where $p_{C}$ is the expected agreement proportion if the two observers are
completely independent. The statistic can be simplified to
\begin{equation}
\kappa = \frac{2 (ad - bc)}{gf + eh}.
\end{equation}
It the two tests are in perfect agreement, $\kappa=1$. If the two agree at the 
level
expected by chance, $\kappa=0$. If the level of agreement is less than one would
obtain by chance alone, $\kappa < 0$.

A formal test of significance of the difference in the probabilities of
for the two tests is obtained using McNemar's test. The null hypothesis is
that the probability of + for test 1 is equal to the probability of + for test
2, or equivalently that the probability of observing a $+-$ is the same as that
of observing $-+$. The normal deviate test statistic is given by
\begin{equation}
z = \frac{b - c}{\sqrt{b + c}}.
\end{equation}

\subsection{Measuring Agreement Between One Observer and a Standard}
Suppose that each of n patients is studied with a diagnostic test and
that the true diagnosis is determined, resulting in the following frequency
table:
\begin{center}\begin{tabular}{cccc}
& & Diagnosis & \\
          &   & + - &  \\
Test      & + & a b & g\\
          & - & c d & h\\
          &   & e f & n\\
\end{tabular}\end{center}
The following measures are frequently used to describe the agreement between
the test and the true diagnosis.  Here $T^{+}$ denotes a positive test, $D^{-}$ 
denotes  no disease, etc.
\begin{center}\begin{tabular}{ccc}
Quantity & Probability Being Estimated & Formula \\ \hline
Correct diagnosis probability & Prob$(T = D)$ &  $\frac{a+d}{n}$ \\
Sensitivity & Prob$(T^{+} | D^{+})$ & $\frac{a}{e}$ \\
Specificity & Prob$(T^{-} | D^{-})$ & $\frac{d}{f}$ \\
Accuracy of a positive test & Prob$(D^{+} | T_{+})$ & $\frac{a}{g}$ \\
Accuracy of a negative test & Prob$(D^{-} | T_{-})$ & $\frac{d}{h}$
\end{tabular}\end{center}
The first and last two measures are usually preferred.
Note that when the disease is very rare or very common, the correct
diagnosis probability will be high by chance alone. Since the sensitivity and
specificity are calculated conditional on the diagnosis, the prevalence of
disease does not directly affect these measures.  But sensitivity and 
specificity will vary with every patient characteristic related to the actual 
ignored severity of disease.  

When estimating any of these quantities, Wilson confidence intervals are useful
adjunct statistics.  A less accurate 0.95 confidence interval is obtained
from $p \pm 1.96\sqrt{\frac{p(1-p)}{n}}$ where $p$ is the proportion
and $m$ is its denominator.


\section{Problems}
\begin{enumerate}
\item Three technicians, using different machines, make 3 readings each. For the
data that follow, calculate estimates of inter- and intra-technician
discrepancy.
\begin{center}\begin{tabular}{ccc}
 & Technician & \\
 1 & 2 & 3 \\ \hline
 Reading & Reading & Reading \\ \hline
 1 2 3 & 1 2 3 & 1 2 3 \\ \hline
18 17 14 & 16 15 16 & 12 15 12 \\
20 21 20 & 14 12    & 13 \\
26 20 23 & 18 20    & 22 24\\
19 17    & 16       & 21 23\\
28 24    & 32 29    & 29 25\\ \hline
\end{tabular}\end{center}

\item Forty-one patients each receive two tests yielding the frequency table
shown below. Calculate a measure of agreement (or disagreement) along with an
associated 0.95 confidence interval. Also calculate a chance-corrected measure
of agreement. Test the null hypothesis that the the tests have the same
probability of being positive and the same probability of being negative. In
other words, test the hypothesis that the chance of observing $+-$ is the same
as observing $-+$.
\begin{center}\begin{tabular}{ccc}
 & & Test 2\\
 & &  + - \\
Test 1 & + & 29 ~8\\
       & - & ~0 ~4\\ \hline
\end{tabular}\end{center}
\end{enumerate}

\section{References}
Landis JR, Koch GG: A review of statistical methods in the analysis of
data arising from observer reliability studies (Part II), \emph{Statistica
Neerlandica} \textbf{29}:151-619 1975.

Landis JR, Koch GG: An application of hierarchical $\kappa$-type statistics
in the assessment of majority agreement among multiple observers.
\emph{Biometrics} \textbf{33}:363-74, 1977.
